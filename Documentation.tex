\documentclass[oneside]{memoir}

\usepackage{fontspec}
% Default font
\setmainfont{Gentium Plus}
% Fallback font for characters unavailable in Gentium
\newfontfamily{\fallbackfont}{Times New Roman}
\DeclareTextFontCommand{\tfb}{\fallbackfont}
% Fallback font for symbols
\newfontfamily{\fallbackfontsymbol}{Segoe UI Symbol}
\DeclareTextFontCommand{\tfbs}{\fallbackfontsymbol}
\usepackage{microtype}
\usepackage{makecell}
\renewcommand\cellalign{tl}
\usepackage{enumitem}
\usepackage{xcolor}

\usepackage{wrapfig}
\setlength\columnsep{20pt}

\usepackage{cprotect}
% Formatting for keys
\newcommand{\key}{\verb}
% Formatting for keys, when \verb is not available
\newcommand{\keynv}{\texttt}
% Formatting for characters produced by Conkey
\newcommand{\out}[1]{\colorbox{gray!20}{\strut{}#1}}
% Formatting for Unicode codepoint
\newcommand{\uni}[1]{\texttt{#1}}
\makeatletter
% \ins{lowercase}{uppercase} gives a list of inputs to a dead key
% \ins*{lower}{upper} formats them in two lines instead of one
\def\ins{\@ifstar\@ins\@@ins}
\newcommand{\@ins}[2]{\keynv{#1}\quad\keynv{#2}}
\newcommand{\@@ins}[2]{\makecell{\keynv{#1}\\\keynv{#2}}}
\makeatother

\setsecnumdepth{subsection}
\settocdepth{subsection}

\setsecheadstyle{\Large\scshape\memRTLraggedright}
\setsubsecheadstyle{\large\scshape\memRTLraggedright}
\setsubsubsecheadstyle{\normalsize\scshape\memRTLraggedright}

\makechapterstyle{sectionsc}{%
  \renewcommand*{\printchaptername}{}
  \renewcommand*{\chapternamenum}{}
  \renewcommand{\chaptitlefont}{\Huge\scshape}
  \renewcommand*{\chapnumfont}{\chaptitlefont}
  \renewcommand*{\printchapternum}{\chapnumfont \thechapter\space}
  \renewcommand*{\afterchapternum}{}
}
\chapterstyle{sectionsc}

\setulmarginsandblock{1in}{*}{*}
\setlrmarginsandblock{1in}{*}{*}
\checkandfixthelayout

\usepackage[colorlinks]{hyperref}

% These are mostly copied from the memoir manual
% with some changes to the font
\pretitle{\begin{center}\HUGE\scshape}
\preauthor{\begin{center}
           \huge \lineskip 0.5em%
           \begin{tabular}[t]{c}}
\predate{\begin{center}\Large}

\title{Guide to Conkey}
\author{Brad Neimann}
\date{\today}

\begin{document}

\maketitle

\tableofcontents

\listoftables

\chapter{How to use this guide}
\label{sec:how_to_use}

\section{Key notation}
\label{sec:key_notation}

This guide uses a notation for key presses based on Emacs key notation:

\begin{itemize}
\item A single letter or symbol such as \key|l| or \key|-| means `press that key'. A special case is \key|SPC|, which means `press the Space bar'.
\item \key|S-key| means `press Shift at the same time as \key|key|'.
\item \key|G-key| means `press AltGr\footnote{The right Alt key} at the same time as \key|key|'.
\item \key|key1 key2| means `press \key|key1|, then release it, then press \key|key2|'.
\end{itemize}

For instance, \key|G-S-- r| means `press AltGr+shift+\key|-| at the same time, then release it, then press \key|r|'.
This key combination results in \out{ṟ} when used with Conkey.

Note that Conkey is designed around the US keyboard layout, rather than the UK keyboard layout.
For instance, \key|\| refers to the key next to \key|]|, rather than the key next to \key|z|.
(For more details on the difference between the two keyboards, see \url{https://en.wikipedia.org/wiki/British_and_American_keyboards#Windows_keyboards}.)

Also note that, although Conkey is designed to use as many mnemonics as possible,
  this notation may sometimes obscure the mnemonicity of certain key sequences.
For instance, it is not immediately obvious that \key|S-1| corresponds to \out{!},
  which makes the mnemonic choice of \key|G-S-1| for \out{¡} less clear.
This particular notation was chosen because it is consistent, easy to learn, and already familiar to many people,
  but it does come at the cost of making such sequences less mnemonic.

\section{Dead keys}
\label{sec:dead_keys}

A \textit{dead key} is a key which, instead of entering a letter or symbol itself, alters the next keystroke.
It does this in a predictable way, by mapping the next keystroke (here referred to as the `input')
  to a specific character (here referred to as the `result').
Dead keys will typically only accept certain inputs,
  with unsupported inputs producing unspecified behaviour\footnotemark.
Conkey relies extensively on dead keys in order to support the entry of as many different characters as possible.
For instance, the dead key \key|G-'| adds an acute accent to the following letter,
  so typing something like \key|G-' a| will generate the letter \out{á}.
\footnotetext{%
On Windows, this will result in a specific character being entered, depending on the dead key.
Press the backspace key to get rid of it.
On Linux, this should just ignore the dead key.%
}

In this guide, the mappings of dead keys are described in a tabular format.
For instance, here is one row from Table~\ref{tab:g-l_mappings},
  which gives a list of mappings for the \key|G-l| dead key:

\medskip

\begin{tabular}{llll}
\toprule
Input & Result & Input for uppercase & Uppercase \\
\midrule
\key|g| & ƣ & \key|S-g| & Ƣ \\
\bottomrule
\end{tabular}

\medskip

This row states that an input of \key|g| will yield a result of \out{ƣ}.
The corresponding uppercase letter \out{Ƣ} can be generated via an input of \key|S-g|.
Thus typing \key|G-l g| will result in \out{ƣ},
  and typing \key|G-l S-g| will result in \out{Ƣ}.
This particular table lists uppercase and lowercase letters on the same row,
  but other tables may not do this (e.g.\ Table~\ref{tab:misc_diacritics_mappings}, which lists diacritics rather than letters).
Refer to the headings of each table for more information.
Occasionally tables will also be divided by horizontal lines into sections;
  this is for readability and does not affect 
Additionally, Table~\ref{tab:dk_for_diacritics}, which lists all the dead keys which implement diacritics,
  uses a different, more compact format, to enable a list of every acceptable input for every dead key;
  refer to Section~\ref{sec:diacritics} for a description of the format of this table.

\section{Formatting conventions}
\label{sec:formatting}

Outside tables, the following formatting conventions are used:

\begin{itemize}[noitemsep]
\item \key|x| refers to the key(s) denoted by \key|x| (see Section~\ref{sec:key_notation} below for details).
\item \out{x} refers to the letter (or symbol) `x', as produced by Conkey.
\end{itemize}

(Inside tables, this formatting gets a bit distracting, and it is not quite as necessary due to the presence of table headings,
  so the above formatting rules are not used nearly as consistently within tables.)

\chapter{Keyboard layout}
\label{sec:keyboard_layout}

\section{Diacritics}
\label{sec:diacritics}

\subsection{Diacritics with precomposed characters}
\label{sec:diacritics_with_precomposed_characters}

\begin{wraptable}[28]{r}{5cm}
\centering
\cprotect\caption{Combining diacritics: mappings for \key|G-]| dead key}
\label{tab:misc_diacritics_mappings}
\begin{tabular}{ll}
\toprule
Input & Result \\
\midrule
\key|S-\| & Vertical line above ◌̍ \\
\key|\|   & Vertical line below ◌̩ \\
\key|S-'| & Double vertical line above ◌̎ \\
\key|,|   & Comma above ◌̓ \\
\key|S-,| & Turned comma above ◌̒ \\
\key|.|   & Reversed comma above  ◌̔ \\
\key|S-.| & Comma above right ◌̕ \\
\key|S-8| & Dot above right ◌͘ \\
\key|`|   & Grave below ◌̖ \\
\key|S-`| & Acute below ◌̗ \\
\midrule
\key|'| & Modifier apostrophe ◌ʼ \\
\key|p| & Modifier prime ◌ʹ \\
\key|S-p| & Modifier double prime ◌ʺ \\
\key|]| & Modifier vertical line ◌ˈ \\
\key|:| & Modifier colon ◌꞉ \\
\key|;| & \makecell{Modifier triangular colon ◌ː\\(i.e.\ \textsc{ipa} length symbol)} \\
\key|-| & Modifier macron ◌ˉ \\
\key|S--| & Modifier low macron ◌ˍ \\
\key|ˊ| & Modifier acute accent ◌ˊ \\
\key|G-ˋ| & Modifier grave accent ◌ˋ \\
\key|)| & Modifier dot vertical bar ◌ꜗ \\
\key|/| & Modifier dot slash ◌ꜘ \\
\key|G--| & Modifier dot horizontal bar ◌ꜙ \\
\key|S-]| & Modifier lower right corner ◌ꜚ \\
\key|G-.| & \makecell{Sinological dot \tfb{◌ꞏ}\\(commonly used as half-colon\\length symbol)} \\
\bottomrule
\end{tabular}
\end{wraptable}

The main mechanism for entering diacritics in Conkey is a number of dead keys,
  enumerated in Table~\ref{tab:dk_for_diacritics}.
These dead keys are used for entering diacritics which have precomposed base+diacritic pairs in Unicode.
Each column of the table contains information about the dead key on that row:
\begin{itemize}
\item `Diacritic' shows the name of the diacritic, as well as a representation of the diacritic itself.
\item `Dead key' shows the location of the dead key on the keyboard.
\item `Acceptable inputs' contains a list of input letters which can be used with this dead key.
  For clarity, every list of inputs in this column has been separated into lower- and uppercase letters (sometimes on two lines).

  This column is present because not all letters can be used with every diacritic,
    as Unicode does not contain precomposed characters for every base+diacritic combination.
  For instance, \out{a̋} cannot be typed using \key|G-S-; a|, since that particular combination is not in Unicode.
\item `Notes' contains any other information which may be relevant to that dead key.
\end{itemize}
For instance, the row for the `ring above' diacritic states that this diacritic is accessed via \key|G-8|,
  and that the available combinations are \key|G-8 a|, \key|G-8 u|, \key|G-8 w|, \key|G-8 y|, \key|G-8 A| and \key|G-8 U|
  (which produce \out{å}, \out{ů}, \out{ẘ}, \out{ẙ}, \out{Å}, \out{Ů} respectively).

Additionally, every one of the above dead keys can be used to insert the appropriate combining diacritic,
  for use in situations where there is no precomposed character.
This can be produced by typing space (here denoted \key|SPC|) after the dead key.
For instance, although there is no precomposed Unicode character for \out{a̋},
  this may still be typed by first typing \key|a|, and then typing \key|G-S-; SPC| to produce the combining double acute accent.


\subsection{Combining diacritics and modifier letters}
\label{sec:combining_diacritics}

In addition to diacritics which come with precomposed characters, Unicode includes many diacritics which exist only as combining diacritics,
  without any corresponding precomposed characters.
These diacritics can be typed using the \key|G-]| dead key, with mappings as shown in Table~\ref{tab:misc_diacritics_mappings}.

Unicode also includes some \textit{modifier letters}.
These are diacritic-like characters which are placed after the base letter rather than above or below it.
Examples include the \textsc{IPA} length symbol, or the apostrophe used to indicate ejectives;
  these both modify a base letter, but are placed after the base like a standalone puncuation character.
Modifier letters can also be typed using the \key|G-]| dead key, using the mappings in Table~\ref{tab:misc_diacritics_mappings}.

\begin{table}
\centerfloat
\caption{Dead keys for diacritics}
\label{tab:dk_for_diacritics}
\renewcommand{\arraystretch}{1.5}
\begin{tabularx}{0.83\paperwidth}{lllX}
\toprule
Diacritic & Dead key & Acceptable inputs & Notes \\
\midrule
Acute ◌́            & \key|G-'|   & \ins{acegiklmnoprsuwyz}{ACEGIKLMNOPRSUWYZ}      &
  \key|G-' G-a|, \key|G-' G-A|, \key|G-' G-o|, \key|G-' G-O| produce \out{ǽ}, \out{Ǽ}, \out{ǿ}, \out{Ǿ} respectively. \newline
  Note that this dead key is also used for single quotes and guillemots. \\
Diaeresis/umlaut ◌̈ & \key|G-S-'| & \ins{aehiotuwxy}{AEHIOUWXY}                     & Note that this dead key is also used for double quotes and guillemots. \\
Diaeresis below ◌̤  & \key|G-3|   & \ins*{u}{U}                                     & \\
Double acute ◌̋     & \key|G-S-;| & \ins*{ou}{OU}                                   & \\
Grave ◌̀            & \key|G-`|   & \ins{aeinouwy}{AEINOUWY}                        & \\
Tilde ◌̃            & \key|G-S-`| & \ins{aeinouvy}{AEINOUVY}                        & \\
Tilde below ◌̰      & \key|G-S-3| & \ins*{elu}{elU}                                 & \\
Caron ◌̌            & \key|G-5|   & \ins{acdeghijklnorstuz}{ACDEGHIKLNORSTUZ}       & \key|G-5 G-z| and \key|G-5 G-Z| produce \out{ǯ} and \out{Ǯ} respectively. \\
Circumflex ◌̂       & \key|G-6|   & \ins{aceghijosuwyz}{ACEGHIJOSUWYZ}              & \\
Circumflex below ◌̭ & \key|G-7|   & \ins*{delntu}{DELNTU}                           & \\
Ring above ◌̊       & \key|G-8|   & \ins*{auwy}{AU}                                 & \\
Breve ◌̆            & \key|G-9|   & \ins*{aegiou}{AEGIOU}                           & \\
Hook above ◌̉       & \key|G-0|   & \ins*{aeiouy}{AEIOUY}                           & \\
Horn ◌̛             & \key|G-S-0| & \ins*{ou}{OU}                                   & \\
Macron ◌̄           & \key|G--|   & \ins{aegiouy}{AEGIOUY}                          &
  \key|G-- G-a| and \key|G-- G-A| produce \out{ǣ} and \out{Ǣ} respectively. \newline
  Note that this dead key is also used for dashes. \\
Line below ◌̱       & \key|G-S--| & \ins{bdhklnrtz}{BDKLNRTZ}                       & \\
Dot below ◌̣        & \key|G-.|   & \ins{abdehiklmnorstuvwyz}{ABDEHIKLMNORSTUVWYZ}  & Note that this dead key is also used for various dot symbols. \\
Dot above ◌̇        & \key|G-S-.| & \ins{abcdefghmnoprstwxyz}{ABCDEFGHIMNOPRSTWXYZ} & Note that due to its position on this keyboard, this dead key is also used to type chevron brackets. \\
Ogonek ◌̨           & \key|G-[|   & \ins*{aeiou}{AEIOU}                             & \\
Comma below ◌̦      & \key|G-,|   & \ins*{st}{ST}                                   & \\
Cedilla ◌̧          & \key|G-S-,| & \ins{cdeghklnrst}{CDEGHKLNRST}                  &
  Some fonts will display these letters (particularly \out{ģ}, \out{ķ}, \out{ļ}, \out{ņ}, \out{ŗ}) with a comma rather than a cedilla.
  See \href{https://en.wikipedia.org/wiki/Cedilla\#Latvian}{Wikipedia} for more information. \newline
  Also note that due to its position on this keyboard, this dead key is also used to type chevron brackets. \\
\bottomrule
\end{tabularx}
\end{table}
\clearpage % to be extra sure this float gets output right at this point

\begin{table}[!b]
\centerfloat
\begin{minipage}{0.16\paperwidth}
\centering
\cprotect\caption{Letters with hooks: mappings for \key|G-S-[| dead key}
\label{tab:hook_mappings}
\begin{tabular}{ll}
\toprule
Input & Result \\
\midrule
\key|b| & ɓ \\
\key|S-b| & Ɓ \\
\key|d| & ɖ \\
\key|S-d| & Ɖ \\
\key|G-d| & ɗ \\
\key|G-S-d| & Ɗ \\
\key|f| & ƒ \\
\key|S-f| & Ƒ \\
\key|g| & ɠ \\
\key|S-g| & Ɠ \\
\key|k| & ƙ \\
\key|S-k| & Ƙ \\
\key|n| & ɲ \\
\key|S-n| & Ɲ \\
\key|p| & ƥ \\
\key|S-p| & Ƥ \\
\key|q| & \tfb{ɋ} \\
\key|S-q| & \tfb{Ɋ} \\
\key|r| & ɽ \\
\key|S-r| & Ɽ \\
\key|t| & ƭ \\
\key|S-t| & Ƭ \\
\key|G-t| & ʈ \\
\key|G-S-t| & Ʈ \\
\key|v| & ʋ \\
\key|S-v| & Ʋ \\
\key|w| & ⱳ \\
\key|S-w| & Ⱳ \\
\key|y| & ƴ \\
\key|S-y| & Ƴ \\
\key|z| & ȥ \\
\key|S-z| & Ȥ \\
\bottomrule
\end{tabular}
\end{minipage}\hfill
\begin{minipage}{0.55\paperwidth}
\centering
\cprotect\caption{Letters with bars: mappings for \key|G-\| and \key|G-S-\| dead keys}
\label{tab:bar_mappings}
\begin{tabular}{lll@{\hspace{1cm}}lll}
\toprule
Input
 & \makecell{Result\\(\keynv{G-\textbackslash})}
 & \makecell{Result\\(\keynv{G-S-\textbackslash})}
& Input
 & \makecell{Result\\(\keynv{G-\textbackslash})}
 & \makecell{Result\\(\keynv{G-S-\textbackslash})} \\
\midrule
\key|a|   & ⱥ &   & \key|o|     & ø & ɵ \\
\key|S-a| & Ⱥ &   & \key|S-o|   & Ø & Ɵ \\
\key|b|   & ƀ &   & \key|p|     & ᵽ & ꝑ \\
\key|S-b| & Ƀ &   & \key|S-p|   & Ᵽ & Ꝑ \\
\key|c|   & ȼ &   & \key|q|     & ꝗ &   \\
\key|S-c| & Ȼ &   & \key|S-q|   & Ꝗ &   \\
\key|d|   & ð & đ & \key|r|     & ɍ &   \\
\key|S-d| & Ð & Đ & \key|S-r|   & Ɍ &   \\
\key|e|   & ɇ &   & \key|t|     & ŧ &   \\
\key|S-e| & Ɇ &   & \key|S-t|   & Ŧ &   \\
\key|g|   & ǥ &   & \key|G-t|   & ꝥ & ꝧ \\
\key|S-g| & Ǥ &   & \key|G-S-t| & Ꝥ & Ꝧ \\
\key|h|   & ħ &   & \key|u|     & ʉ & \\
\key|S-h| & Ħ &   & \key|S-u|   & Ʉ & \\
\key|i|   & ɨ &   & \key|y|     & ɏ & \\
\key|S-i| & Ɨ &   & \key|S-y|   & Ɏ & \\
\key|j|   & ɉ & ɟ & \key|z|     & ƶ & \\
\key|S-j| & Ɉ &   & \key|S-z|   & Ƶ & \\
\key|k|   & ꝁ &   & \key|G-j|   & ƛ & \\
\key|S-k| & Ꝁ &   &             &   & \\
\key|l|   & ł & ƚ &             &   & \\
\key|S-l| & Ł & Ƚ &             &   & \\
\key|G-l| &   & ⱡ &             &   & \\
\key|G-S-l| & & Ⱡ &             &   & \\
\bottomrule
\end{tabular}
\end{minipage}
\end{table}

\subsection{Diacritics spanning two letters}
\label{sec:diacritics_spanning_two_letters}

Certain diacritics span two letters rather than one --- for example, the tie used for \textsc{ipa} diphthongs such as \out{a͡i}, or the line above the Menominee grapheme \out{a͞e}.
Such diacritics can be typed using the dead key \key|G-2|:

\begin{itemize}[noitemsep]
\item \key|G-2 G-9| yields the double inverted breve below \out{◌͜◌}
\item \key|G-2 G-0| yields the double inverted breve above \out{◌͝◌}
\item \key|G-2 G--| yields the double macron above \out{◌͞◌}
\item \key|G-2 S--| yields the double macron below \out{◌͟◌}
\item \key|G-2 S-`| yields the double tilde above \out{◌͠◌}
\item \key|G-2 G-6| yields the double tie above \out{◌͡◌}
\end{itemize}

These diacritics must be typed between the two individual letters they span;
  for instance, \out{a͞e} is typed as \key|a G-2 G-- e|.

\subsection{Letters containing diacritics (hooks and bars)}
\label{sec:letters_containing_diacritics}

Certain letters are treated in Conkey as if they contain a diacritic,
  despite the fact that the `diacritic' is joined to the base letter and does not have a fixed location.
These include letters with hooks (e.g.\ \out{ɓ} and \out{ƒ}) and bars (e.g.\ \out{ð} and \out{ʉ}).
Hooks are produced via the \key|G-S-[| dead key, with mappings as in Table~\ref{tab:hook_mappings},
  while bars are produced via the \key|G-\| and \key|G-S-\| dead keys, with mappings as in Table~\ref{tab:bar_mappings}.
(Two dead keys are necessary for bars as some barred characters have two variations.)

Conkey also includes four combining diacritics for bars.
\out{◌̵} and \out{◌̷} are produced by \key|C-\ SPC| and \key|C-\ \| respectively,
  while the longer bars \out{◌̶} and \out{◌̸} are produced by \key|C-S-\ SPC| and \key|C-S-\ \| respectively.

\section{Letters}
\label{sec:letters}

\subsection{Non-English letters}
\label{sec:non-eng_letters}

\begin{table}
\centering
\begin{minipage}{0.4\linewidth}
\centering
\caption{Latin-script letters accessed via a single key}
\label{tab:single_key}
\begin{tabular}{llll}
\toprule
Input & Result & Input for uppercase & Uppercase \\
\midrule
\key|G-a| & æ & \key|G-S-a| & Æ \\
\key|G-e| & ɛ & \key|G-S-e| & Ɛ \\
\key|G-r| & ə & \key|G-S-r| & Ə \\
\key|G-g| & ɣ & \key|G-S-g| & Ɣ \\
\key|G-i| & ı &             &  \\
\key|G-l| & ɬ & \key|G-S-l| & Ɬ \\
\key|G-n| & ŋ & \key|G-S-n| & Ŋ \\
\key|G-o| & ɔ & \key|G-S-o| & Ɔ \\
\key|G-k| & œ & \key|G-S-k| & Œ \\
\key|G-s| & ß & \key|G-S-s| & ẞ \\
\key|G-t| & þ & \key|G-S-t| & Þ \\
\key|G-u| & ɯ & \key|G-S-u| & Ɯ \\
\key|G-v| & ʌ & \key|G-S-v| & Ʌ \\
\key|G-z| & ʒ & \key|G-S-z| & Ʒ \\
\key|G-x| & ʔ &             &  \\
\key|G-c| & ʕ &             &  \\
\key|G-q| & ɂ & \key|G-S-q| & Ɂ \\
\key|G-b| & ꞌ & \key|G-S-b| & Ꞌ \\
\key|G-S-s| & ʻ &             &  \\
\bottomrule
\end{tabular}
\end{minipage}\hfill
\begin{minipage}{0.4\linewidth}
\centering
\cprotect\caption{Latin-script letters accessed via \key|G-d|}
\label{tab:letters_g-d}
\begin{tabular}{llll}
\toprule
Input & Result & Input for uppercase & Uppercase \\
\midrule
\key|G-d a| & ɑ       & \key|G-d S-a| & Ɑ \\
\key|G-d h| & ḫ       & \key|G-d S-h| & Ḫ \\
\key|G-d b| & \tfb{ꞵ} & \key|G-d S-b| & \tfb{Ꞵ} \\
\key|G-d e| & ǝ       & \key|G-d S-e| & Ǝ \\
\key|G-d i| & ɩ       & \key|G-d S-i| & Ɩ \\
\key|G-d w| & \tfb{ꞷ} & \key|G-d S-w| & \tfb{Ꞷ} \\
\key|G-d u| & ʊ       & \key|G-d S-u| & Ʊ \\
\key|G-d s| & ʃ       & \key|G-d S-s| & Ʃ \\
\key|G-d z| & ƹ       & \key|G-d S-z| & Ƹ \\
\key|G-d x| & ꜣ       &               & Ꜣ \\
\key|G-d c| & ꜥ       &               & Ꜥ \\
\bottomrule
\end{tabular}
\end{minipage}

\begin{minipage}{0.4\linewidth}
\centering
\cprotect\caption{Latin-script letters accessed via \key|G-m|}
\label{tab:letters_g-m}
\begin{tabular}{llll}
\toprule
Input & Result & Input for uppercase & Uppercase \\
\midrule
\key|G-m k| & ĸ &               &  \\
\key|G-m n| & ƞ & \key|G-m S-n| & Ƞ \\
\key|G-m o| & ꝏ & \key|G-m S-o| & Ꝏ \\
\key|G-m u| & ȣ & \key|G-m S-u| & Ȣ \\
\key|G-m 3| & ꜫ & \key|G-m S-3| & Ꜫ \\
\key|G-m 4| & ꜭ & \key|G-m S-4| & Ꜭ \\
\key|G-m 5| & ꜯ & \key|G-m S-5| & Ꜯ \\
\bottomrule
\end{tabular}
\end{minipage}\hfill
\begin{minipage}{0.4\linewidth}
\centering
\cprotect\caption{Latin-script letters accessed via \key|G-f|}
\label{tab:letters_g-f}
\begin{tabular}{llll}
\toprule
Input & Result & Input for uppercase & Uppercase \\
\midrule
\key|G-f b| & ƃ & \key|G-f S-b| & Ƃ \\
\key|G-f d| & ƌ & \key|G-f S-d| & Ƌ \\
\key|G-f g| & ƣ & \key|G-f S-g| & Ƣ \\
\key|G-f h| & ꜧ & \key|G-f S-h| & Ꜧ \\
\key|G-f j| & ƕ & \key|G-f S-j| & Ƕ \\
\key|G-f n| & ꞑ & \key|G-f S-n| & Ꞑ \\
\key|G-f s| & ſ &               &  \\
\key|G-f w| & ƿ & \key|G-f S-w| & Ƿ \\
\key|G-f y| & ȝ & \key|G-f S-y| & Ȝ \\
\key|G-f '| & ь & \key|G-f S-'| & Ь \\
\bottomrule
\end{tabular}
\end{minipage}
\end{table}

Conkey includes many Latin-script letters which are not used in English.
These letters may be accessed using one-~and two-key sequences as follows:

\begin{itemize}
\item
  The most common letters may be accessed using only a single key, with an AltGr modifier.
  This category includes letters currently used by European languages (e.g.~\out{ß}),
    particularly widespread non-European letters (e.g.~\out{ɛ}),
    and letters which are widely used by conlangers (e.g.~\out{ɬ}).
  These letters are listed in Table~\ref{tab:single_key}.
\item
  Letters used most commonly in African languages (e.g.~\out{Ǝ}),
    as well as those used for the transcription of Semitic and Ancient~Egyptian (e.g.~\out{ḫ} and~\out{ꜣ}),
    may be accessed by two-key sequences starting with \key|G-d|.
  These letters are listed in Table~\ref{tab:letters_g-d}.
\item
  Letters formerly used in orthographies from Asia and the former~USSR (e.g.~\out{ƃ} and~\out{ƣ}),
    as well as letters formerly used in Europe (e.g.~\out{ſ}),
    may be accessed by two-key sequences starting with \key|G-f|.
  These letters are listed in Table~\ref{tab:letters_g-f}.
\item
  Letters used in American orthographies e.g.~\out{ꝏ})
    may be accessed by two-key sequences starting with \key|G-m|.
  These letters are listed in Table~\ref{tab:letters_g-m}.
\end{itemize}

% These letters are shown in Table~\ref{tab:latin_letters}, along with the keys to produce each letter and its capital (if any).

% Since there is not enough space on the keyboard to include every Latin-script letter,
%   two dead keys are also used, which map keystrokes to (semi-arbitrary-chosen) other letters.
% For instance, \key|G-l g| returns \out{ƣ}, and \key|G-l S-'| returns \out{Ꜣ}.
% The main dead key used for this purpose is \key|G-l|;
%   its mappings are shown in Table~\ref{tab:g-l_mappings}.
% There are also a small number of letters which are produced via the dead key \key|G-o|;
%   the mappings of this key are shown as part of Table~\ref{tab:latin_letters}.

\subsection{Click letters}
\label{sec:clicks}

Click letters can be produced using the \key|G-p| dead key.
For convenience, there are several ways of producing each click letter:

\begin{itemize}[noitemsep]
\item \out{ʘ} can be produced via \key|G-p S-2|
\item \out{ǀ} can be produced via \key|G-p S-\|, \key|G-p \| or \key|G-p c|
\item \out{ǃ} can be produced via \key|G-p S-1| or \key|G-p q|
\item \out{ǂ} can be produced via \key|G-p =| or \key|G-p v|
\item \out{ǁ} can be produced via \key|G-p x|
\end{itemize}

\subsection{Greek letters}
\label{sec:greek_letters}

\begin{table}
\centering
\caption{Greek letters: mappings for \keynv{G-j}}
\label{tab:greek_letters}
\begin{tabular}{llllllll}
\toprule
Input & Result & Input for uppercase & Uppercase & Input & Result & Input for uppercase & Uppercase \\
\midrule
\key|a| & α & \key|S-a| & Α & \key|q| & ξ & \key|S-q| & Ξ \\
\key|b| & β & \key|S-b| & Β & \key|o| & ο & \key|S-o| & Ο \\
\key|g| & γ & \key|S-g| & Γ & \key|p| & π & \key|S-p| & Π \\
\key|d| & δ & \key|S-d| & Δ & \key|r| & ρ & \key|S-r| & Ρ \\
\key|e| & ε & \key|S-e| & Ε & \key|s| & σ & \key|S-s| & Σ \\
\key|z| & ζ & \key|S-z| & Ζ & \key|c| & ς & N/A & N/A \\
\key|h| & η & \key|S-h| & Η & \key|t| & τ & \key|S-t| & Τ \\
\key|j| & θ & \key|S-j| & Θ & \key|u| & υ & \key|S-u| & Υ \\
\key|i| & ι & \key|S-i| & Ι & \key|f| & φ & \key|S-f| & Φ \\
\key|k| & κ & \key|S-k| & Κ & \key|x| & χ & \key|S-x| & Χ \\
\key|l| & λ & \key|S-l| & Λ & \key|v| & ψ & \key|S-v| & Ψ \\
\key|m| & μ & \key|S-m| & Μ & \key|w| & ω & \key|S-w| & Ω \\
\key|n| & ν & \key|S-n| & Ν & & & & \\
\bottomrule
\end{tabular}
\end{table}

\begin{table}[b]
\centering
\caption{Zhuang tone letters: mappings for \keynv{G-1}}
\label{tab:zhuang_tones}
\centering
\begin{tabular}{llll}
\toprule
Input & Result & Input for uppercase & Uppercase \\
\midrule
\key|2| & ƨ & \key|S-2| & Ƨ \\
\key|3| & з & \key|S-3| & З \\
\key|4| & ч & \key|S-4| & Ч \\
\key|5| & ƽ & \key|S-5| & Ƽ \\
\key|6| & ƅ & \key|S-6| & Ƅ \\
\bottomrule
\end{tabular}
\end{table}

Greek letters can be produced using the \key|G-j| dead key.
The mappings for this dead key are shown in Table~\ref{tab:greek_letters}.

\subsection{Superscripts and subscripts}
\label{sec:superscripts-subscripts}

Superscript letters and symbols can be typed using \key|G-w <char>|, where \key|<char>| is one of the following:
\begin{itemize}[noitemsep]
\item Any lowercase English letter other than \key|q|
\item Any uppercase English letter other than \key|S-c|, \key|S-f|, \key|S-q|, \key|S-s|, \key|S-x|, \key|S-y| or \key|S-z|
\item Any number
\item \key|+|, \key|-| or \key|G-j| (which produces a superscript \out{ᶿ})
\end{itemize}
For instance, a superscript \out{ʷ} can be produced by \key|G-w w|.

Similarly, subscripts may be typed using \key|G-S-w <char>|, where \key|<char>| is one of the following:
\begin{itemize}[noitemsep]
\item Any number
\item A lowercase letter in the following list: \texttt{aehijklmnoprstuvx}
\item \key|+| or \key|-|
\end{itemize}

\subsection{Zhuang tone letters}
\label{sec:zhuang_tones}


Zhuang tone letters can be accessed via the \key|G-1| dead key.
The mappings are shown in Table~\ref{tab:zhuang_tones}.

\begin{table}[t]
\caption{Quotation marks and guillemots: mappings for \keynv{G-'} and \keynv{G-S-'}}
\label{tab:quotes_guillemots}
\centering
\begin{tabular}{lll}
\toprule
Input & \makecell{Result\\\keynv{G-'}} & \makecell{Result\\\keynv{G-S-'}} \\
\midrule
\key|S-9| & ‘ & “ \\
\key|S-0| & ’ & ” \\
\key|,|   & ‚ & „ \\
\key|.|   & ‛ & ‟ \\
\key|S-,| & ‹ & « \\
\key|S-.| & › & » \\
\bottomrule
\end{tabular}
\end{table}

\section{Symbols}
\label{sec:symbols}

\subsection{Quotation marks, guillemots and chevrons}
\label{sec:quotation_marks_guillemots_chevrons}

Quotation marks and guillemots can be typed using the dead keys
  \key|G-'| (for single marks) and \key|G-S-'| (for double marks).
These mappings are listed in Table~\ref{tab:quotes_guillemots}.

Although, strictly speaking, the chevrons \out{⟨} and \out{⟩} are brackets rather than quotes or guillemots,
  Conkey allows these to be typed as well using \key|G-S-, G-S-,| and \key|G-S-. G-S-.| respectively.

\subsection{Dashes}
\label{sec:dashes}

Dashes can be produced using the \key|G--| dead key.
\key|G-- m|, \key|G-- n|, \key|G-- -| and~\key|G-- =|
  produce an em~dash, en~dash, hyphen and minus~sign respectively.
(Plain \key|-| produces a \textit{hyphen-minus},
  an \textsc{ascii} symbol which can act as both a hyphen and a minus
  but is not recommended in professionally-produced text.)

\subsection{Dots}
\label{sec:dots}

Various different kinds of dots can be produced using the \key|G-.| dead key:

\begin{itemize}[noitemsep]
\item \key|G-. .| produces an ellipsis \out{…}
\item \key|G-. S-\| produces a middot or interpunct \out{·}
\item \key|G-. S-0| produces a bullet point \out{•}
\item \key|G-. 0| produces a degree sign \out{°}
\item \key|G-. S-2| produces a diacritic carrier sign \out{◌}
  (used as a base to display diacritics, as in Table~\ref{tab:dk_for_diacritics})
\end{itemize}

\begin{table}
\centerfloat
\begin{minipage}{0.25\paperwidth}
\caption{Symbols and punctuation marks}
\label{tab:symbols_punctuation}
\centering
\begin{tabular}{ll}
\toprule
Key sequence & Result \\
\midrule
\key|G-S-1|                    & ¡ \\
\key|G-S-/|                    & ¿ \\
\key|G-S-2 S-1|                & \tfb{‽} \\
\key|G-S-2 S-n|                & № \\
\key|G-S-2 s|                  & § \\
\key|G-S-2 1|                  & † \\
\key|G-S-2 2|                  & ‡ \\
\key|G-S-2 S-\|                & ‖ \\
\midrule
\key|G-S-2 S-c|                & © \\
\key|G-S-2 S-p|                & ℗ \\
\key|G-S-2 S-r|                & ® \\
\key|G-S-2 G-s|                & \tfb{℠} \\
\key|G-S-2 S-t|                & ™ \\
\midrule
\key|G-S-2 y|                  & \tfbs{✓} \\
\key|G-S-2 n|                  & \tfbs{✗} \\
\key|G-S-2 G-S-y|              & \tfbs{✔} \\
\key|G-S-2 G-S-n|              & \tfbs{✘} \\
\key|G-S-2 0|                  & \tfbs{☐} \\
\key|G-S-2 G-y|                & \tfbs{☑} \\
\key|G-S-2 G-n|                & \tfbs{☒} \\
\midrule
\key|G-S-2 c|                  & c (i.e.\ cent symbol) \\
\key|G-S-2 G-c|                & ¢ \\
\key|G-S-2 e|                  & € \\
\makecell{\keynv{G-S-2 l}\\\quad or \keynv{G-S-2 S-3}} & £ \\
\key|G-S-2 o|                  & ¤ \\
\key|G-S-2 r|                  & ₹ \\
\key|G-S-2 S-s|                & ₪ \\
\key|G-S-2 S-y|                & ¥ \\
\bottomrule
\end{tabular}
\end{minipage}\hfill
\begin{minipage}{0.5\paperwidth}
\caption{Spaces: mappings for \keynv{G-SPC} dead key}
\label{tab:spaces}
\centering
\begin{tabular}{lll}
\toprule
Input     & Space produced                      & Code point \\
\midrule
\key|t|   & Tab                                 & \uni{U+0009} \\
\key|~|   & Non-breaking space                  & \uni{U+00a0} \\
\key|n|   & En space                            & \uni{U+2002} \\
\key|m|   & Em space                            & \uni{U+2003} \\
\key|b|   & Three-per-em space                  & \uni{U+2004} \\
\key|v|   & Four-per-em space                   & \uni{U+2005} \\
\key|c|   & Six-per-em space                    & \uni{U+2006} \\
\key|1|   & Figure space                        & \uni{U+2007} \\
\key|.|   & Punctuation space                   & \uni{U+2008} \\
\key|,|   & Thin space                          & \uni{U+2009} \\
\key|S-\| & Hair space                          & \uni{U+200a} \\
\key|S-=| & Medium mathematical space           & \uni{U+205f} \\
\key|S-1| & Narrow no-break space               & \uni{U+202f} \\
\key|0|   & Zero width space                    & \uni{U+200b} \\
\key|S-9| & Zero width non-joiner               & \uni{U+200c} \\
\key|S-0| & Zero width joiner                   & \uni{U+200d} \\
\key|[|   & Open box \out{␣} (symbol for space) & \uni{U+2423} \\
\key|]|
  & \makecell{Shouldered open box \out{\tfbs{⍽}}\\\quad(symbol for non-breaking space)}
  & \uni{U+237d} \\
\bottomrule&&
\end{tabular}
\end{minipage}
\end{table}

\subsection{Other textual symbols and punctuation marks (including currencies and IP marks)}
\label{sec:symbols_punctuation}

Aside from the punctuation marks already covered above,
  Conkey can produce a wide variety of other punctuation marks and symbols,
  mostly via the \key|G-S-2| dead key.
These can be accessed via the key sequences in Table~\ref{tab:symbols_punctuation}.

\subsection{Spaces}
\label{sec:spaces}

Conkey allows many different varieties of space to be produced via the \key|G-SPC| dead key
  (i.e.\ AltGr+spacebar).
The types of space which can be produced using this dead key,
  along with their Unicode code points,
  are shown in Table~\ref{tab:spaces}.

\begin{table}
\centering
\begin{minipage}{0.4\linewidth}
\caption{Arrows: mappings for \keynv{G-/} dead key}
\label{tab:arrows}
\centering
\begin{tabular}{l>{\fallbackfontsymbol}l}
\toprule
Input & \multicolumn{1}{l}{Result} \\
\midrule
\key|w| & ↑ \\
\key|a| & ← \\
\key|s| & ↓ \\
\key|d| & → \\
\key|z| & ↔ \\
\key|x| & ↕ \\
\key|c| & ⇐ \\
\key|v| & ⇒ \\
\key|e| & ⇈ \\
\key|r| & ⇉ \\
\key|f| & ⇊ \\
\key|b| & ⇋ \\
\key|n| & ⇌ \\
\key|m| & ↦ \\
\key|t| & ⤚ \\
\key|y| & ⤙ \\
\key|g| & ⤜ \\
\key|h| & ⤛ \\
\key|u| & ↣ \\
\key|i| & ↢ \\
\key|j| & ⇔ \\
\key|k| & ⇕ \\
\bottomrule&
\end{tabular}
\end{minipage}\hfill
\begin{minipage}{0.4\linewidth}
\centering
\caption{Mathematical and related symbols: mappings for \keynv{G-=} dead key}
\label{tab:math}
\begin{tabular}{l >{\fallbackfontsymbol}l @{\hspace{1.5cm}} l >{\fallbackfontsymbol}l}
\toprule
Input & \multicolumn{1}{l}{Result} & Input & \multicolumn{1}{l}{Result} \\
\midrule
\key|x|     & × & \key|S-[| & ⊆ \\
\key|'|     & ⋅ & \key|S-]| & ⊇ \\
\key|/|     & ÷ & \key|\|   & ∖ \\
\key|-|     & ± & \key|S-;| & ∷ \\
\key|s|     & √ & \key|S-8| & ★ \\
\key|=|     & ≡ & \key|b|   & ⊥ \\
\key|S-1|   & ≠ & \key|S-9| & ⦇ \\
\key|G-S-1| & ≢ & \key|S-0| & ⦈ \\
\key|x|     & ≈ & \key|[|   & ⟦ \\
\key|S-,|   & ≤ & \key|]|   & ⟧ \\
\key|S-.|   & ≥ & \key|f|   & ≫ \\
\key|S-5|   & ‰ & \key|d|   & ≪ \\
\key|1|     & ′ & \key|h|   & ⋙ \\
\key|2|     & ″ & \key|g|   & ⋘ \\
\key|3|     & ‴ & \key|8|   & ⁂ \\
\key|.|     & ∘ & \key|S-\| & ⫴ \\
\key|S-a|   & ∀ & \key|S-3| & ⧻ \\
\key|S-`|   & ¬ & \key|S-=| & ⧺ \\
\key|S-6|   & ∧ & \key|i|   & ‼ \\
\key|v|     & ∨ & \key|S-2| & ⊛ \\
\key|S-7|   & ∴ & \key|a|   & ⊕ \\
\key|0|     & ∅ & \key|S-c| & ℂ \\
\key|e|     & ∈ & \key|S-n| & ℕ \\
\key|S-e|   & ∉ & \key|S-q| & ℚ \\
\key|S-u|   & ∪ & \key|S-r| & ℝ \\
\key|S-i|   & ∩ & \key|S-z| & ℤ \\
\bottomrule
\end{tabular}
\end{minipage}
\end{table}

\subsection{Arrows}
\label{sec:arrows}

Arrows are produced by the \key|G-/| dead key.
Arrows are arranged in a roughly mnemonic way around the keyboard:
  for instance, the most common arrows \out{↑}, \out{←}, \out{↓} and~\out{→}
  are produced by typing \key|G-/ w|, \key|G-/ a|, \key|G-/ s| and \key|G-/ d| respectively.
The full list of mappings can be found in Table~\ref{tab:arrows}.

\subsection{Mathematical and related symbols}
\label{sec:mathematical_and_related_symbols}

Mathematical symbols --- as well as other, related symbols, such as those used in Haskell --- can be produced by the \key|G-=| dead key.
In more detail, this dead key allows the production of:
\begin{itemize}[noitemsep]
\item Symbols from basic algebra (\out{×}, \out{≠}, \out{≥} etc.)
\item Symbols from logic (\out{\tfbs{∀}}, \out{\tfbs{∧}} etc.)
\item Symbols from set theory (\out{\tfbs{∈}}, \out{\tfbs{∩}} etc.)
\item Unicode versions of Haskell syntax and operators (\out{\tfbs{∷}}, \out{\tfbs{⊕}}, \out{\tfbs{‼}} etc.)
\item Blackboard bold letters (\out{\tfbs{ℝ}}, \out{\tfbs{ℕ}} etc.)
\end{itemize}
The mappings for this dead key are shown in Table~\ref{tab:math}.

\end{document}

% Local Variables:
% TeX-engine: xetex
% End:
