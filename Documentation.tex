\documentclass[oneside]{memoir}

\usepackage{fontspec}
% Default font
\setmainfont{Gentium Plus}
% Fallback font for characters unavailable in Gentium
\newfontfamily{\fallbackfont}{Times New Roman}
\DeclareTextFontCommand{\tfb}{\fallbackfont}
% Fallback font for symbols
\newfontfamily{\fallbackfontsymbol}{Segoe UI Symbol}
\DeclareTextFontCommand{\tfbs}{\fallbackfontsymbol}
\usepackage{microtype}
\usepackage{makecell}
\renewcommand\cellalign{tl}
\usepackage{enumitem}
\usepackage{xcolor}
\usepackage{pifont}
% from https://tex.stackexchange.com/a/42620/155372
\newcommand{\cmark}{\ding{51}}
\newcommand{\xmark}{\ding{55}}

\usepackage{wrapfig}
\setlength\columnsep{20pt}

\usepackage{cprotect}
% Formatting for keys
\newcommand{\key}{\verb}
% Formatting for keys, when \verb is not available
\newcommand{\keynv}{\texttt}
% Formatting for characters produced by Conkey
\newcommand{\out}[1]{\colorbox{gray!20}{\strut{}#1}}
% Formatting for Unicode codepoint
\newcommand{\uni}[1]{\texttt{#1}}
\makeatletter
% \ins{lowercase}{uppercase} gives a list of inputs to a dead key
% \ins*{lower}{upper} formats them in two lines instead of one
\def\ins{\@ifstar\@ins\@@ins}
\newcommand{\@ins}[2]{\keynv{#1}\quad\keynv{#2}}
\newcommand{\@@ins}[2]{\makecell{\keynv{#1}\\\keynv{#2}}}
\makeatother

\setsecnumdepth{subsection}
\settocdepth{subsection}

\setsecheadstyle{\Large\scshape\memRTLraggedright}
\setsubsecheadstyle{\large\scshape\memRTLraggedright}
\setsubsubsecheadstyle{\normalsize\scshape\memRTLraggedright}

\makechapterstyle{sectionsc}{%
  \renewcommand*{\printchaptername}{}
  \renewcommand*{\chapternamenum}{}
  \renewcommand{\chaptitlefont}{\Huge\scshape}
  \renewcommand*{\chapnumfont}{\chaptitlefont}
  \renewcommand*{\printchapternum}{\chapnumfont \thechapter\space}
  \renewcommand*{\afterchapternum}{}
}
\chapterstyle{sectionsc}

\setulmarginsandblock{1in}{*}{*}
\setlrmarginsandblock{1in}{*}{*}
\checkandfixthelayout

\usepackage[colorlinks]{hyperref}

% These are mostly copied from the memoir manual
% with some changes to the font
\pretitle{\begin{center}\HUGE\scshape}
\preauthor{\begin{center}
           \huge \lineskip 0.5em%
           \begin{tabular}[t]{c}}
\predate{\begin{center}\Large}

\title{Guide to Conkey}
\author{Brad Neimann}
\date{\today}

\begin{document}

\maketitle

\tableofcontents

\listoftables

\chapter{How to use this guide}
\label{sec:how_to_use}

\section{Key notation}
\label{sec:key_notation}

This guide uses a notation for key presses based on Emacs key notation:

\begin{itemize}
\item A single letter or symbol such as \key|l| or \key|-| means `press that key'. A special case is \key|SPC|, which means `press the Space bar'.
\item
  An uppercase letter, or a symbol which is entered using the Shift key, means `press the appropriate key while holding Shift'.
  For instance, \key|A| means `hold Shift while pressing \key|a|', and \key|!| means `hold Shift while pressing \key|1|'.
\item \key|S-key| means `press Shift at the same time as \key|key|'\footnotemark.
\item \key|G-key| means `press AltGr\footnote{The right Alt key} at the same time as \key|key|'.
\item \key|key1 key2| means `press \key|key1|, then release it, then press \key|key2|'.
\end{itemize}

\footnotetext{%
This particular piece of notation is a bit redundant (e.g.~\keynv{S-a} is the same as \keynv{A})
  but is useful for making certain keys easier to remember:
  for instance, the cedilla is \keynv{G-<}, but is slightly easier to remember as \keynv{G-S-,} (the comma looks like a cedilla).%
Throughout this guide, the two notations will be used interchangeably.
}

For instance, \key|G-_ r| means `press AltGr+shift+\key|-| at the same time, then release them, then press \key|r|'.
This key combination results in \out{ṟ} when used with Conkey.

Note that Conkey is designed around the US keyboard layout, rather than the UK keyboard layout.
For instance, \key|\| refers to the key next to \key|]|, rather than the key next to \key|z|.
(For more details on the difference between the two keyboards, see \url{https://en.wikipedia.org/wiki/British_and_American_keyboards#Windows_keyboards}.)

\section{Dead keys}
\label{sec:dead_keys}

A \textit{dead key} is a key which, instead of entering a letter or symbol itself, changes something about the next keystroke.
Usually a dead key will add a diacritic to the next letter; Conkey uses many dead keys for this purpose.
For instance, the dead key \key|G-'| adds an acute accent to the following letter,
  so typing something like \key|G-' a| will generate the letter~\out{á}.

Conkey also uses some dead keys which modify the following letter with a hook or bar.
For instance, the dead key \key|G-{| adds a hook to the following letter,
  so typing something like \key|G-{ b| will generate the letter~\out{ɓ}.

Note that dead keys will typically only accept certain inputs,
  with unsupported inputs producing unspecified behaviour\footnotemark.
The allowed inputs for each dead key will be specified in tabular format when each key is described.
\footnotetext{%
On Windows, this will result in an unwanted character being entered.
Press the backspace key to get rid of it.
On Linux, this should just do nothing.%
}

\section{Key sequences}
\label{sec:key_seqs}

To allow the input of punctuation and non-English letters, Conkey makes extensive use of \textit{key sequences}.
This term refers to sequences of two keys, which when typed in sequence produce only one letter.
For instance, the left quote \out{‘} is produced by the key sequence \key|G-' (|, while the letter \out{ƣ} is produced by the key sequence \key|G-f g|.
(Note that if you press an incorrect key sequence, you will either get no result or an incorrect output;
  it is fine to simply delete any incorrect output which you accidentally produced.)

Key sequences are used primarily for reasons of space: there is simply not enough space on the keyboard to fit every letter and symbol supported by Conkey,
  so key sequences are the only way for Conkey to support as many letters and symbols as it does.
Additionally, key sequences can have a mnemonic role: for instance, all key sequences starting with \key|G-p| produce click letters,
  which is preferable to having the click letters scattered semi-randomly around the keyboard.

When using key sequences to input letters, one important detail to be aware of is capitalisation.
To capitalise a letter which is produced by a key sequence, the Shift key should only be held during the last key in the sequence.
For instance, the lowercase letter \out{ƣ} is produced by the key sequence \key|G-f g|,
  while the capital letter \out{Ƣ} is produced by the key sequence \key|G-f G|,
  rather than \key|G-F g| or \key|G-F G|.

\section{Formatting conventions}
\label{sec:formatting}

Outside tables, the following formatting conventions are used:

\begin{itemize}[noitemsep]
\item \key|x| (in a monospace font) refers to a key which you need to press (using the key notation described in Section~\ref{sec:key_notation} above).
\item \out{x} (with a grey background) )refers to a letter or symbol which is produced by Conkey.
\end{itemize}

Additionally, tables are used when listing the various letters and symbols of Conkey.
These tables will look something like this:

\medskip

\begin{tabular}{lll}
\toprule
Input & Result & Uppercase? \\
\midrule
\key|G-a|   & æ & \cmark \\
\key|G-d i| & ɩ & \cmark \\
\key|G-d k| & ĸ & \xmark \\
\bottomrule
\end{tabular}

\medskip

The first column, marked `Input', shows a key or key sequence.
When this key or sequence is pressed, it produces the letter shown in the second column, `Result'.
For instance, pressing \key|G-a| results in \out{æ}.
The third column, marked `Uppercase?', contains a check~mark if that letter corresponds to an uppercase letter,
  and a cross~mark if that letter does not correspond to any uppercase letter.
For instance, pressing \key|G-d I| results in uppercase \out{Ɩ},
  but pressing \key|G-d K| does not give any sensible output.
(Recall that uppercase letters are entered by holding the Shift key;
  for a key sequence, Shift must be held only during the last key of the sequence.)
The `Uppercase?' column may be missing for tables which list symbols rather than letters, as symbols generally lack uppercases.

\chapter{Keyboard layout}
\label{sec:keyboard_layout}

\section{Diacritics}
\label{sec:diacritics}

\subsection{Diacritics with precomposed characters}
\label{sec:diacritics_with_precomposed_characters}

\begin{wraptable}[30]{r}{5cm}
\centering
\cprotect\caption{Combining diacritics and modifier letters}
\label{tab:misc_diacritics_mappings}
\begin{tabular}{ll}
\toprule
Input & Result \\
\midrule
\key!G-] |!   & Vertical line above ◌̍ \\
\key|G-] \|   & Vertical line below ◌̩ \\
\key|G-] "|   & Double vertical line above ◌̎ \\
\key|G-] ,|   & Comma above ◌̓ \\
\key|G-] <|   & Turned comma above ◌̒ \\
\key|G-] .|   & Reversed comma above  ◌̔ \\
\key|G-] >|   & Comma above right ◌̕ \\
\key|G-] *|   & Dot above right ◌͘ \\
\key|G-] `|   & Grave below ◌̖ \\
\key|G-] S-`| & Acute below ◌̗ \\
\midrule
\key|G-] '|   & Modifier apostrophe ◌ʼ \\
\key|G-] p|   & Modifier prime ◌ʹ \\
\key|G-] P|   & Modifier double prime ◌ʺ \\
\key!G-] |!   & Modifier vertical line ◌ˈ \\
\key|G-] :|   & Modifier colon ◌꞉ \\
\key|G-] ;|   & \makecell{Modifier triangular colon ◌ː\\(i.e.\ \textsc{ipa} length symbol)} \\
\key|G-] =|   & Modifier equals sign ◌꞊ \\
\key|G-] -|   & Modifier macron ◌ˉ \\
\key|G-] _|   & Modifier low macron ◌ˍ \\
\key|G-] G-'| & Modifier acute accent ◌ˊ \\
\key|G-] G-ˋ| & Modifier grave accent ◌ˋ \\
\key|G-] )|   & Modifier dot vertical bar ◌ꜗ \\
\key|G-] /|   & Modifier dot slash ◌ꜘ \\
\key|G-] G--| & Modifier dot horizontal bar ◌ꜙ \\
\key|G-] }|   & Modifier lower right corner ◌ꜚ \\
\key|G-] G-.| & \makecell{Sinological dot \tfb{◌ꞏ}\\(commonly used as half-colon\\length symbol)} \\
\bottomrule
\end{tabular}
\end{wraptable}

The main mechanism for entering diacritics in Conkey is a number of dead keys,
  enumerated in Table~\ref{tab:dk_for_diacritics}.
Each column of the table contains information about the dead key on that row:
\begin{itemize}
\item `Diacritic' shows the name of the diacritic, as well as a representation of the diacritic itself.
\item `Dead key' shows the location of the dead key on the keyboard.
\item `Acceptable inputs' contains a list of input letters which can be used with this dead key.\footnotemark
  For clarity, every list of inputs in this column has been separated into lower- and uppercase letters (sometimes on two lines).

  This column is required because not all letter+diacritic combinations are present in Unicode;
    the missing combination thus cannot be typed in the same way.
  For instance, \out{a̋} cannot be typed using \key|G-: a|, since that particular combination is not in Unicode.
\item `Notes' contains any other information which may be relevant to that dead key.
\end{itemize}
For instance, the row for the `ring above' diacritic states that this diacritic is accessed via \key|G-8|,
  and that the available combinations are \key|G-8 a|, \key|G-8 u|, \key|G-8 w|, \key|G-8 y|, \key|G-8 A| and \key|G-8 U|
  (which produce \out{å}, \out{ů}, \out{ẘ}, \out{ẙ}, \out{Å}, \out{Ů} respectively).

Additionally, every one of the above dead keys can be used to insert the corresponding combining diacritic.
This is a form of the diacritic which acts as an independent character and combines with the previous letter,
  rather than simply being a part of the next letter.
Combining diacritics may be inserted by pressing the Space bar immediately after the appropriate dead key.
Combining diacritics are useful when Unicode does not include a precomposed base+diacritic combination;
  for instance, although \out{a̋} is not present as an independent letter,
  it may still be entered by first typing \key|a|, and then typing \key|G-: SPC| to produce the combining double acute accent.
(Note that if you want to delete a letter+diacritic inserted in this manner,
 you will need to press Backspace twice rather than once:
  once to delete the diacritic, and then another time to delete the letter.)

\subsection{Combining diacritics and modifier letters}
\label{sec:combining_diacritics}

In addition to diacritics which come with precomposed characters, Unicode includes many diacritics which exist only as combining diacritics,
  without any corresponding precomposed characters.
These diacritics can be entered using key sequences starting with \key|G-]|, as shown in Table~\ref{tab:misc_diacritics_mappings}.

Unicode also includes some \textit{modifier letters}.
These are diacritic-like characters which are placed after the base letter rather than above or below it.
Examples include the \textsc{IPA} length symbol, or the apostrophe used to indicate ejectives;
  these both modify a base letter, but are placed after the base like a standalone puncuation character.
(Occasionally modifier letters are also used as independent letters;
  for instance, when an apostrophe is used as a letter for the glottal stop,
  the recommendation is to use the modifier apostrophe rather than the punctuation apostrophe.)
Modifier letters can also be entered using key sequences starting with \key|G-]|, as shown in Table~\ref{tab:misc_diacritics_mappings}.

\begin{table}
\centerfloat
\caption{Dead keys for diacritics}
\label{tab:dk_for_diacritics}
\renewcommand{\arraystretch}{1.5}
\begin{tabularx}{0.83\paperwidth}{lllX}
\toprule
Diacritic & Dead key & Acceptable inputs & Notes \\
\midrule
Acute ◌́            & \key|G-'|   & \ins{acegiklmnoprsuwyz}{ACEGIKLMNOPRSUWYZ}      &
  \key|G-' G-a|, \key|G-' G-A|, \key|G-' G-o|, \key|G-' G-O| produce \out{ǽ}, \out{Ǽ}, \out{ǿ}, \out{Ǿ} respectively. \\
Diaeresis/umlaut ◌̈ & \key|G-"|   & \ins{aehiotuwxy}{AEHIOUWXY}                     & \\
Diaeresis below ◌̤  & \key|G-3|   & \ins*{u}{U}                                     & \\
Double acute ◌̋     & \key|G-:|   & \ins*{ou}{OU}                                   & \\
Grave ◌̀            & \key|G-`|   & \ins{aeinouwy}{AEINOUWY}                        & \\
Tilde ◌̃            & \key|G-~|   & \ins{aeinouvy}{AEINOUVY}                        & \\
Tilde below ◌̰      & \key|G-#|   & \ins*{eiu}{EIU}                                 & \\
Caron ◌̌            & \key|G-5|   & \ins{acdeghijklnorstuz}{ACDEGHIKLNORSTUZ}       & \key|G-5 G-z| and \key|G-5 G-Z| produce \out{ǯ} and \out{Ǯ} respectively. \\
Circumflex ◌̂       & \key|G-6|   & \ins{aceghijosuwyz}{ACEGHIJOSUWYZ}              & \\
Circumflex below ◌̭ & \key|G-7|   & \ins*{delntu}{DELNTU}                           & \\
Ring above ◌̊       & \key|G-8|   & \ins*{auwy}{AU}                                 & \\
Breve ◌̆            & \key|G-9|   & \ins*{aegiou}{AEGIOU}                           & \\
Hook above ◌̉       & \key|G-0|   & \ins*{aeiouy}{AEIOUY}                           & \\
Horn ◌̛             & \key|G-)|   & \ins*{ou}{OU}                                   & \\
Macron ◌̄           & \key|G--|   & \ins{aegiouy}{AEGIOUY}                          &
  \key|G-- G-a| and \key|G-- G-A| produce \out{ǣ} and \out{Ǣ} respectively. \\
Line below ◌̱       & \key|G-_|   & \ins{bdhklnrtz}{BDKLNRTZ}                       & \\
Dot below ◌̣        & \key|G-.|   & \ins{abdehiklmnorstuvwyz}{ABDEHIKLMNORSTUVWYZ}  & \\
Dot above ◌̇        & \key|G-S-.| & \ins{abcdefghmnoprstwxyz}{ABCDEFGHIMNOPRSTWXYZ} & \\
Ogonek ◌̨           & \key|G-[|   & \ins*{aeiou}{AEIOU}                             & \\
Comma below ◌̦      & \key|G-,|   & \ins*{st}{ST}                                   & \\
Cedilla ◌̧          & \key|G-S-,| & \ins{cdeghklnrst}{CDEGHKLNRST}                  &
  Some fonts will display these letters (particularly \out{ģ}, \out{ķ}, \out{ļ}, \out{ņ}, \out{ŗ}) with a comma rather than a cedilla.
  See \href{https://en.wikipedia.org/wiki/Cedilla\#Latvian}{Wikipedia} for more information. \\
\bottomrule
\end{tabularx}
\end{table}
\clearpage % to be extra sure this float gets output right at this point

\begin{table}[!b]
\centerfloat
\begin{minipage}{0.16\paperwidth}
\centering
\caption{Letters with hooks: mappings for \keynv{G-\{} dead key}
\label{tab:hook_mappings}
\begin{tabular}{lll}
\toprule
Input & Result & Uppercase? \\
\midrule
\key|b|   & ɓ       & \cmark \\
\key|d|   & ɖ       & \cmark \\
\key|G-d| & ɗ       & \cmark \\
\key|f|   & ƒ       & \cmark \\
\key|g|   & ɠ       & \cmark \\
\key|k|   & ƙ       & \cmark \\
\key|n|   & ɲ       & \cmark \\
\key|p|   & ƥ       & \cmark \\
\key|q|   & \tfb{ɋ} & \cmark \\
\key|r|   & ɽ       & \cmark \\
\key|t|   & ƭ       & \cmark \\
\key|G-t| & ʈ       & \cmark \\
\key|v|   & ʋ       & \cmark \\
\key|w|   & ⱳ       & \cmark \\
\key|y|   & ƴ       & \cmark \\
\key|z|   & ȥ       & \cmark \\
\bottomrule
\end{tabular}
\end{minipage}\hfill
\begin{minipage}{0.55\paperwidth}
\centering
\cprotect\caption{Letters with bars: mappings for \key|G-\| and \key!G-|! dead keys}
\label{tab:bar_mappings}
\begin{tabular}{lllll}
\toprule
Input
 & \makecell{Result\\(\keynv{G-\textbackslash})}
 & Uppercase?
 & \makecell{Result\\(\keynv{G-|})}
 & Uppercase? \\
\midrule
\key|a|   & ⱥ & \cmark &   & \\
\key|b|   & ƀ & \cmark &   & \\
\key|c|   & ȼ & \cmark &   & \\
\key|d|   & ð & \cmark & đ & \cmark \\
\key|e|   & ɇ & \cmark &   & \\
\key|g|   & ǥ & \cmark &   & \\
\key|h|   & ħ & \cmark &   & \\
\key|i|   & ɨ & \cmark &   & \\
\key|j|   & ɉ & \cmark & ɟ & \xmark \\
\key|k|   & ꝁ & \cmark &   & \\
\key|l|   & ł & \cmark & ƚ & \cmark \\
\key|G-l| &   &        & ⱡ & \cmark \\
\key|o|   & ø & \cmark & ɵ & \cmark \\
\key|p|   & ᵽ & \cmark & ꝑ & \cmark \\
\key|q|   & ꝗ & \cmark &   & \\
\key|r|   & ɍ & \cmark &   & \\
\key|t|   & ŧ & \cmark &   & \\
\key|G-t| & ꝥ & \cmark & ꝧ & \cmark \\
\key|u|   & ʉ & \cmark &   & \\
\key|y|   & ɏ & \cmark &   & \\
\key|z|   & ƶ & \cmark &   & \\
\key|G-j| & ƛ & \xmark &   & \\
\bottomrule
\end{tabular}
\end{minipage}
\end{table}

\subsection{Diacritics spanning two letters}
\label{sec:diacritics_spanning_two_letters}

Certain diacritics span two letters rather than one --- for example, the tie used for \textsc{ipa} diphthongs such as \out{a͡i}, or the line above the Menominee grapheme \out{a͞e}.
Such diacritics may be entered using key sequences starting with \key|G-2|:

\begin{itemize}[noitemsep]
\item \key|G-2 G-9| yields the double inverted breve below \out{◌͜◌}
\item \key|G-2 G-0| yields the double inverted breve above \out{◌͝◌}
\item \key|G-2 G--| yields the double macron above \out{◌͞◌}
\item \key|G-2 _| yields the double macron below \out{◌͟◌}
\item \key|G-2 ~| yields the double tilde above \out{◌͠◌}
\item \key|G-2 G-6| yields the double tie above \out{◌͡◌}
\end{itemize}

These diacritics must be typed between the two individual letters they span;
  for instance, \out{a͞e} is entered as \key|a G-2 G-- e|.

\subsection{Letters containing diacritics (hooks and bars)}
\label{sec:letters_containing_diacritics}

Certain letters are treated in Conkey as if they contain a diacritic,
  despite the fact that the `diacritic' is joined to the base letter and does not have a fixed location.
These include letters with hooks (e.g.\ \out{ɓ} and \out{ƒ}) and bars (e.g.\ \out{ð} and \out{ʉ}).
Hooks are produced via the \key|G-{| dead key, with mappings as in Table~\ref{tab:hook_mappings},
  while bars are produced via the \key|G-\| and \key!G-|! dead keys, with mappings as in Table~\ref{tab:bar_mappings}.
(Two dead keys are necessary for bars as some barred characters have two variations.)
For instance, \out{ɓ} is produced by typing \key|G-{ b|,
  \out{ø} is produced by typing \key|G-\ o|,
  and \out{ɵ} is produced by typing \key!G-| o!.

Conkey also includes four combining diacritics for bars.
\out{◌̵} and \out{◌̷} are produced by \key|C-\ SPC| and \key|C-\ \| respectively,
  while the longer bars \out{◌̶} and \out{◌̸} are produced by \key!C-| SPC! and \key!C-| \! respectively.

\section{Letters}
\label{sec:letters}

\subsection{Non-English letters}
\label{sec:non-eng_letters}

\begin{table}
\centering
\begin{minipage}{0.4\linewidth}
\centering
\caption{Latin-script letters accessed via a single key}
\label{tab:single_key}
\begin{tabular}{lll}
\toprule
Input & Result & Uppercase? \\
\midrule
\key|G-a| & æ & \cmark \\
\key|G-e| & ɛ & \cmark \\
\key|G-r| & ə & \cmark \\
\key|G-g| & ɣ & \cmark \\
\key|G-i| & ı & \xmark \\
\key|G-l| & ɬ & \cmark \\
\key|G-n| & ŋ & \cmark \\
\key|G-o| & ɔ & \cmark \\
\key|G-k| & œ & \cmark \\
\key|G-s| & ß & \cmark \\
\key|G-t| & þ & \cmark \\
\key|G-u| & ɯ & \cmark \\
\key|G-v| & ʌ & \cmark \\
\key|G-z| & ʒ & \cmark \\
\key|G-x| & ʔ & \xmark \\
\key|G-c| & ʕ & \xmark \\
\key|G-q| & ɂ & \cmark \\
\key|G-b| & ꞌ & \cmark \\
\key|G-X| & ʻ & \xmark \\
\bottomrule
\end{tabular}
\end{minipage}\hfill
\begin{minipage}{0.4\linewidth}
\centering
\cprotect\caption{Latin-script letters accessed via \key|G-d|}
\label{tab:letters_g-d}
\begin{tabular}{lll}
\toprule
Input & Result & Uppercase? \\
\midrule
\key|G-d a| & ɑ       & \cmark \\
\key|G-d b| & \tfb{ꞵ} & \cmark \\
\key|G-d e| & ǝ       & \cmark \\
\key|G-d h| & ḫ       & \cmark \\
\key|G-d i| & ɩ       & \cmark \\
\key|G-d k| & ĸ       & \xmark \\
\key|G-d n| & ƞ       & \cmark \\
\key|G-d s| & ʃ       & \cmark \\
\key|G-d o| & ꝏ       & \cmark \\
\key|G-d u| & ʊ       & \cmark \\
\key|G-d w| & \tfb{ꞷ} & \cmark \\
\key|G-d z| & ƹ       & \cmark \\
\key|G-d 3| & ꜫ       & \cmark \\
\key|G-d 4| & ꜭ       & \cmark \\
\key|G-d 5| & ꜯ       & \cmark \\
\key|G-d 8| & ȣ       & \cmark \\
\key|G-d c| & ꜥ       & \cmark \\
\key|G-d x| & ꜣ       & \cmark \\
\bottomrule
\end{tabular}
\end{minipage}

\begin{minipage}{0.4\linewidth}
\centering
\cprotect\caption{Latin-script letters accessed via \key|G-f|}
\label{tab:letters_g-f}
\begin{tabular}{lll}
\toprule
Input & Result & Uppercase? \\
\midrule
\key|G-f b| & ƃ & \cmark \\
\key|G-f d| & ƌ & \cmark \\
\key|G-f g| & ƣ & \cmark \\
\key|G-f h| & ꜧ & \cmark \\
\key|G-f j| & ƕ & \cmark \\
\key|G-f n| & ꞑ & \cmark \\
\key|G-f s| & ſ & \xmark \\
\key|G-f w| & ƿ & \cmark \\
\key|G-f y| & ȝ & \cmark \\
\key|G-f '| & ь & \cmark \\
\bottomrule
\end{tabular}
\end{minipage}
\end{table}

Conkey includes many Latin-script letters which are not used in English.
These letters may be accessed using one-~and two-key sequences as follows:

\begin{itemize}
\item
  The most common letters may be accessed using only a single key, with an AltGr modifier.
  This category includes letters currently used by European languages (e.g.~\out{ß}),
    particularly widespread non-European letters (e.g.~\out{ɛ}),
    and letters which are widely used by conlangers (e.g.~\out{ɬ}).
  These letters are listed in Table~\ref{tab:single_key}.
\item
  Letters used in current and former orthographies of Eurasia (e.g.~\out{ƃ} and~\out{ƣ})
    --- including historic European letters (e.g.~\out{ſ}), but excluding letters used primarily for the transcription of Semitic languages (e.g.~\out{ḫ}) ---
    may be accessed by two-key sequences starting with \key|G-f|.
  These letters are listed in Table~\ref{tab:letters_g-f}.
\item
  All other letters may be accessed by two-key sequences starting with \key|G-d|.
  These letters are listed in Table~\ref{tab:letters_g-d}.
\end{itemize}

\subsection{Click letters}
\label{sec:clicks}

Click letters can be produced using key sequences starting with \key|G-p|:

\begin{itemize}[noitemsep]
\item \out{ʘ} can be produced using \key|G-p @|
\item \out{ǀ} can be produced using \key!G-p |! or \key|G-p \| or \key|G-p c|
\item \out{ǃ} can be produced using \key|G-p !| or \key|G-p q|
\item \out{ǂ} can be produced using \key|G-p =| or \key|G-p v|
\item \out{ǁ} can be produced using \key|G-p x|
\end{itemize}

\subsection{Greek letters}
\label{sec:greek_letters}

\begin{table}
\centering
\caption{Greek letters}
\label{tab:greek_letters}
\begin{tabular}{llllll}
\toprule
Input & Result & Uppercase? & Input & Result & Uppercase? \\
\midrule
\key|G-j a| & α & \cmark & \key|G-j q| & ξ & \cmark \\
\key|G-j b| & β & \cmark & \key|G-j o| & ο & \cmark \\
\key|G-j g| & γ & \cmark & \key|G-j p| & π & \cmark \\
\key|G-j d| & δ & \cmark & \key|G-j r| & ρ & \cmark \\
\key|G-j e| & ε & \cmark & \key|G-j s| & σ & \cmark \\
\key|G-j z| & ζ & \cmark & \key|G-j c| & ς & \xmark \\
\key|G-j h| & η & \cmark & \key|G-j t| & τ & \cmark \\
\key|G-j j| & θ & \cmark & \key|G-j u| & υ & \cmark \\
\key|G-j i| & ι & \cmark & \key|G-j f| & φ & \cmark \\
\key|G-j k| & κ & \cmark & \key|G-j x| & χ & \cmark \\
\key|G-j l| & λ & \cmark & \key|G-j v| & ψ & \cmark \\
\key|G-j m| & μ & \cmark & \key|G-j w| & ω & \cmark \\
\key|G-j n| & ν & \cmark & & & \\
\bottomrule
\end{tabular}
\end{table}

\begin{table}[b]
\centering
\caption{Zhuang tone letters}
\label{tab:zhuang_tones}
\centering
\begin{tabular}{lll}
\toprule
Input & Result & Uppercase? \\
\midrule
\key|G-1 2| & ƨ & \cmark \\
\key|G-1 3| & з & \cmark \\
\key|G-1 4| & ч & \cmark \\
\key|G-1 5| & ƽ & \cmark \\
\key|G-1 6| & ƅ & \cmark \\
\bottomrule
\end{tabular}
\end{table}

Greek letters can be produced using key sequences starting with \key|G-j|.
These key sequences are shown in Table~\ref{tab:greek_letters}.

\subsection{Superscripts and subscripts}
\label{sec:superscripts-subscripts}

Superscript letters and symbols can be entered using key sequences of the form \key|G-w <char>|,
  where \key|<char>| is one of the following:
\begin{itemize}[noitemsep]
\item Any lowercase English letter other than \key|q|
\item Any uppercase English letter other than \key|C|, \key|F|, \key|Q|, \key|S|, \key|X|, \key|Y| or \key|Z|
\item Any number
\item \key|+|, \key|-| or \key|G-j| (which produces a superscript \out{ᶿ})
\end{itemize}
For instance, a superscript \out{ʷ} can be produced by \key|G-w w|.

Similarly, subscripts may be entered using key sequences of the form \key|G-W <char>|,
  where \key|<char>| is one of the following:
\begin{itemize}[noitemsep]
\item Any number
\item Any lowercase letter in the following list: \texttt{aehijklmnoprstuvx}
\item \key|+| or \key|-|
\end{itemize}

\subsection{Zhuang tone letters}
\label{sec:zhuang_tones}

Zhuang tone letters can be accessed via key sequences starting with \key|G-1|, shown in Table~\ref{tab:zhuang_tones}.

\begin{table}[t]
\caption{Quotation marks and guillemots}
\label{tab:quotes_guillemots}
\centering
\begin{tabular}{lll}
\toprule
& \multicolumn{2}{c}{First key} \\
\cmidrule(l){2-3}
Second key & \keynv{G-' …} & \keynv{G-" …} \\
\midrule
\keynv{… (} & ‘ & “ \\
\keynv{… )} & ’ & ” \\
\keynv{… ,} & ‚ & „ \\
\keynv{… .} & ‛ & ‟ \\
\keynv{… <} & ‹ & « \\
\keynv{… >} & › & » \\
\bottomrule
\end{tabular}
\end{table}

\section{Symbols}
\label{sec:symbols}

\subsection{Quotation marks, guillemots and chevrons}
\label{sec:quotation_marks_guillemots_chevrons}

Quotation marks, guillemots can be entered using key sequences starting with
  \key|G-'| (for single marks) and \key|G-"| (for double marks).
These key sequences are listed in Table~\ref{tab:quotes_guillemots}.
For instance, \key|G-" )| produces the double quotation mark~\out{”}, and \key|G-' <| produces the single guillemot~\out{‹}.
Additionally, the chevrons \out{⟨} and \out{⟩} may be entered using the key sequences \key|G-< G-<| and \key|G-> G->| respectively.

\subsection{Dashes}
\label{sec:dashes}

Dashes can be produced using key sequences starting with \key|G--|.
In particular \key|G-- m|, \key|G-- n|, \key|G-- -| and~\key|G-- =|
  produce an em~dash, en~dash, hyphen and minus~sign respectively.
(Plain \key|-| produces a \textit{hyphen-minus},
  an \textsc{ascii} symbol which can act as both a hyphen and a minus.)
It is also possible to enter double~hyphens:
  \key|G-- S-=| produces a plain~double~hyphen \out{\tfb{⹀}}, and \key|G-- /| produces an oblique~double~hyphen \out{\tfb{⸗}}.

\subsection{Dots}
\label{sec:dots}

Various different kinds of dots can be produced using key sequences starting with \key|G-.|:

\begin{itemize}[noitemsep]
\item \key|G-. .| produces an ellipsis \out{…}
\item \key!G-. |! produces a middot or interpunct \out{·}
\item \key|G-. )| produces a bullet point \out{•}
\item \key|G-. 0| produces a degree sign \out{°}
\item \key|G-. @| produces a diacritic carrier sign \out{◌}
  (used as a base to display diacritics, as in Table~\ref{tab:dk_for_diacritics})
\end{itemize}

\begin{table}
\centerfloat
\begin{minipage}{0.25\paperwidth}
\caption{Symbols and punctuation marks}
\label{tab:symbols_punctuation}
\centering
\begin{tabular}{ll}
\toprule
Input & Result \\
\midrule
\key|G-!|     & ¡ \\
\key|G-?|     & ¿ \\
\key|G-@ !|   & \tfb{‽} \\
\key|G-@ N|   & № \\
\key|G-@ s|   & § \\
\key|G-@ 1|   & † \\
\key|G-@ 2|   & ‡ \\
\key!G-@ |!   & ‖ \\
\midrule
\key|G-@ C|   & © \\
\key|G-@ P|   & ℗ \\
\key|G-@ R|   & ® \\
\key|G-@ G-s| & \tfb{℠} \\
\key|G-@ T|   & ™ \\
\midrule
\key|G-@ y|   & \tfbs{✓} \\
\key|G-@ n|   & \tfbs{✗} \\
\key|G-@ G-Y| & \tfbs{✔} \\
\key|G-@ G-N| & \tfbs{✘} \\
\key|G-@ 0|   & \tfbs{☐} \\
\key|G-@ G-y| & \tfbs{☑} \\
\key|G-@ G-n| & \tfbs{☒} \\
\midrule
\key|G-@ c|   & c (i.e.\ cent symbol) \\
\key|G-@ G-c| & ¢ \\
\key|G-@ e|   & € \\
% \makecell{\keynv{G-@ l}\\\quad or \keynv{G-@ #}} & £ \\
\key|G-@ o|   & ¤ \\
\key|G-@ r|   & ₹ \\
\key|G-@ S|   & ₪ \\
\key|G-@ Y|   & ¥ \\
\bottomrule
\end{tabular}
\end{minipage}\hfill
\begin{minipage}{0.5\paperwidth}
\caption{Spaces}
\label{tab:spaces}
\centering
\begin{tabular}{lll}
\toprule
Input     & Space produced                      & Code point \\
\midrule
\key|G-SPC t| & Tab                                 & \uni{U+0009} \\
\key|G-SPC ~| & Non-breaking space                  & \uni{U+00a0} \\
\key|G-SPC n| & En space                            & \uni{U+2002} \\
\key|G-SPC m| & Em space                            & \uni{U+2003} \\
\key|G-SPC b| & Three-per-em space                  & \uni{U+2004} \\
\key|G-SPC v| & Four-per-em space                   & \uni{U+2005} \\
\key|G-SPC c| & Six-per-em space                    & \uni{U+2006} \\
\key|G-SPC 1| & Figure space                        & \uni{U+2007} \\
\key|G-SPC .| & Punctuation space                   & \uni{U+2008} \\
\key|G-SPC ,| & Thin space                          & \uni{U+2009} \\
\key!G-SPC |! & Hair space                          & \uni{U+200a} \\
\key|G-SPC +| & Medium mathematical space           & \uni{U+205f} \\
\key|G-SPC !| & Narrow no-break space               & \uni{U+202f} \\
\key|G-SPC 0| & Zero width space                    & \uni{U+200b} \\
\key|G-SPC (| & Zero width non-joiner               & \uni{U+200c} \\
\key|G-SPC )| & Zero width joiner                   & \uni{U+200d} \\
\key|G-SPC [| & Open box \out{␣} (symbol for space) & \uni{U+2423} \\
\key|G-SPC ]|
  & \makecell{Shouldered open box \out{\tfbs{⍽}}\\\quad(symbol for non-breaking space)}
  & \uni{U+237d} \\
\bottomrule&&
\end{tabular}
\end{minipage}
\end{table}

\subsection{Other textual symbols and punctuation marks (including currencies and IP marks)}
\label{sec:symbols_punctuation}

Aside from the punctuation marks already covered above,
  Conkey can produce a wide variety of other punctuation marks and symbols,
  mostly via key sequences starting with \key|G-@|.
These punctuation marks are listed in Table~\ref{tab:symbols_punctuation}.

\subsection{Spaces}
\label{sec:spaces}

Conkey allows many different varieties of space to be produced via key sequences starting with \key|G-SPC| (i.e.\ AltGr+spacebar).
The types of space which can be produced using this dead key, along with their Unicode code points,
  are shown in Table~\ref{tab:spaces}.

\begin{table}
\centering
\begin{minipage}{0.4\linewidth}
\caption{Arrows}
\label{tab:arrows}
\centering
\begin{tabular}{l>{\fallbackfontsymbol}l}
\toprule
Input & \multicolumn{1}{l}{Result} \\
\midrule
\key|G-/ w| & ↑ \\
\key|G-/ a| & ← \\
\key|G-/ s| & ↓ \\
\key|G-/ d| & → \\
\key|G-/ z| & ↔ \\
\key|G-/ x| & ↕ \\
\key|G-/ c| & ⇐ \\
\key|G-/ v| & ⇒ \\
\key|G-/ e| & ⇈ \\
\key|G-/ r| & ⇉ \\
\key|G-/ f| & ⇊ \\
\key|G-/ b| & ⇋ \\
\key|G-/ n| & ⇌ \\
\key|G-/ m| & ↦ \\
\key|G-/ t| & ⤚ \\
\key|G-/ y| & ⤙ \\
\key|G-/ g| & ⤜ \\
\key|G-/ h| & ⤛ \\
\key|G-/ u| & ↣ \\
\key|G-/ i| & ↢ \\
\key|G-/ j| & ⇔ \\
\key|G-/ k| & ⇕ \\
\bottomrule&
\end{tabular}
\end{minipage}\hfill
\begin{minipage}{0.4\linewidth}
\centering
\caption{Mathematical and related symbols}
\label{tab:math}
\begin{tabular}{l >{\fallbackfontsymbol}l @{\hspace{1.5cm}} l >{\fallbackfontsymbol}l}
\toprule
Input & \multicolumn{1}{l}{Result} & Input & \multicolumn{1}{l}{Result} \\
\midrule
\key|G-= x|   & × & \key|G-= {| & ⊆ \\
\key|G-= '|   & ⋅ & \key|G-= }| & ⊇ \\
\key|G-= /|   & ÷ & \key|G-= \| & ∖ \\
\key|G-= -|   & ± & \key|G-= :| & ∷ \\
\key|G-= s|   & √ & \key|G-= *| & ★ \\
\key|G-= =|   & ≡ & \key|G-= b| & ⊥ \\
\key|G-= !|   & ≠ & \key|G-= (| & ⦇ \\
\key|G-= G-!| & ≢ & \key|G-= )| & ⦈ \\
\key|G-= G-x| & ≈ & \key|G-= [| & ⟦ \\
\key|G-= <|   & ≤ & \key|G-= ]| & ⟧ \\
\key|G-= >|   & ≥ & \key|G-= f| & ≫ \\
\key|G-= %|   & ‰ & \key|G-= d| & ≪ \\
\key|G-= 1|   & ′ & \key|G-= h| & ⋙ \\
\key|G-= 2|   & ″ & \key|G-= g| & ⋘ \\
\key|G-= 3|   & ‴ & \key|G-= 8| & ⁂ \\
\key|G-= .|   & ∘ & \key!G-= |! & ⫴ \\
\key|G-= A|   & ∀ & \key|G-= #| & ⧻ \\
\key|G-= ~|   & ¬ & \key|G-= +| & ⧺ \\
\key|G-= ^|   & ∧ & \key|G-= i| & ‼ \\
\key|G-= v|   & ∨ & \key|G-= @| & ⊛ \\
\key|G-= &|   & ∴ & \key|G-= a| & ⊕ \\
\key|G-= 0|   & ∅ & \key|G-= C| & ℂ \\
\key|G-= e|   & ∈ & \key|G-= N| & ℕ \\
\key|G-= E|   & ∉ & \key|G-= Q| & ℚ \\
\key|G-= U|   & ∪ & \key|G-= R| & ℝ \\
\key|G-= I|   & ∩ & \key|G-= Z| & ℤ \\
\bottomrule
\end{tabular}
\end{minipage}
\end{table}

\subsection{Arrows}
\label{sec:arrows}

Arrows are produced by key sequences starting with \key|G-/|.
Arrows are arranged in a roughly mnemonic way around the keyboard:
  for instance, the most common arrows \out{↑}, \out{←}, \out{↓} and~\out{→}
  are produced by typing \key|G-/ w|, \key|G-/ a|, \key|G-/ s| and \key|G-/ d| respectively.
The full list of key sequences can be found in Table~\ref{tab:arrows}.

\subsection{Mathematical and related symbols}
\label{sec:mathematical_and_related_symbols}

Mathematical symbols --- as well as other, related symbols, such as those used in Haskell --- can be produced by key sequences starting with \key|G-=|.
In more detail, this dead key allows the production of:
\begin{itemize}[noitemsep]
\item Symbols from basic algebra (\out{×}, \out{≠}, \out{≥} etc.)
\item Symbols from logic (\out{\tfbs{∀}}, \out{\tfbs{∧}} etc.)
\item Symbols from set theory (\out{\tfbs{∈}}, \out{\tfbs{∩}} etc.)
\item Unicode versions of Haskell syntax and operators (\out{\tfbs{∷}}, \out{\tfbs{⊕}}, \out{\tfbs{‼}} etc.)
\item Blackboard bold letters (\out{\tfbs{ℝ}}, \out{\tfbs{ℕ}} etc.)
\end{itemize}
The full list of key sequences is given in Table~\ref{tab:math}.

\end{document}

% Local Variables:
% TeX-engine: xetex
% End:
